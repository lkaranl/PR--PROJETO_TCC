%%%%%%%%%%%%%%%%%%%%%%%%%%%%%%%%%%%%%%%%%%%%%%%%%%%%%%%%%%%%%%%%%%%%%%%%%%%%%%%%
% 
% UNEMAT-BBG-Tex (v1.0)
% https://github.com/lkaranl/PRE-PROJETO-TCC-LATEX
%
% MODELO DO PRÉ-PROJETO DE TCC DA UNEMAT
% 
%%%%%%%%%%%%%%%%%%%%%%%%%%%%%%%%%%%%%%%%%%%%%%%%%%%%%%%%%%%%%%%%%%%%%%%%%%%%%%%%
% 
% MIT License
% 
% Copyright (c) 2019 Karan Luciano Silva
% 
% Permission is hereby granted, free of charge, to any person obtaining a copy
% of this software and associated documentation files (the "Software"), to deal
% in the Software without restriction, including without limitation the rights
% to use, copy, modify, merge, publish, distribute, sublicense, and/or sell
% copies of the Software, and to permit persons to whom the Software is
% furnished to do so, subject to the following conditions:
% 
% The above copyright notice and this permission notice shall be included in all
% copies or substantial portions of the Software.
% 
% THE SOFTWARE IS PROVIDED "AS IS", WITHOUT WARRANTY OF ANY KIND, EXPRESS OR
% IMPLIED, INCLUDING BUT NOT LIMITED TO THE WARRANTIES OF MERCHANTABILITY,
% FITNESS FOR A PARTICULAR PURPOSE AND NONINFRINGEMENT. IN NO EVENT SHALL THE
% AUTHORS OR COPYRIGHT HOLDERS BE LIABLE FOR ANY CLAIM, DAMAGES OR OTHER
% LIABILITY, WHETHER IN AN ACTION OF CONTRACT, TORT OR OTHERWISE, ARISING FROM,
% OUT OF OR IN CONNECTION WITH THE SOFTWARE OR THE USE OR OTHER DEALINGS IN THE
% SOFTWARE.
%
%%%%%%%%%%%%%%%%%%%%%%%%%%%%%%%%%%%%%%%%%%%%%%%%%%%%%%%%%%%%%%%%%%%%%%%%%%%%%%%%

\documentclass{unemat-tex}

%%%%%%%%%%%%%%%%%%%%%%%%%%%%%%%%%%%%%%%%%%%%%%%%%%%%%%%%%%%%%%%%%%%%%%%%%%%%%%%%
% Inclusão de pacotes
%%%%%%%%%%%%%%%%%%%%%%%%%%%%%%%%%%%%%%%%%%%%%%%%%%%%%%%%%%%%%%%%%%%%%%%%%%%%%%%%

\usepackage{lmodern}			% Usa a fonte Latin Modern
\usepackage[T1]{fontenc}		% Selecao de codigos de fonte.
\usepackage[utf8]{inputenc}		% Codificacao do documento (conversão automática dos acentos)
\usepackage{indentfirst}		% Indenta o primeiro parágrafo de cada seção.
\usepackage{color}				% Controle das cores
\usepackage{graphicx}			% Inclusão de gráficos
\usepackage{microtype} 			% para melhorias de justificação
\usepackage{scalefnt}
\usepackage{balance}
\usepackage{float}
\usepackage{placeins}
\usepackage{hyperref}
\usepackage{mathptmx}
\usepackage{times}
\usepackage{textcomp}
\usepackage{gensymb} % degree. ie 360º
\usepackage{amsmath}
\usepackage{comment}
\usepackage{algorithm}
\usepackage{algorithmic}
\usepackage{multirow}
\usepackage{caption}
\usepackage{hyperref}

% Uso da fonte Arial (UNEMAT)
\usepackage{helvet}
\renewcommand{\familydefault}{\sfdefault}

% Incluir pacotes adicionais, caso sejam necessários ....

\usepackage{hyperref} % Required for hyperlinks
\hypersetup{hidelinks,colorlinks,breaklinks=true,urlcolor=color2,citecolor=color1,linkcolor=color1,bookmarksopen=false,pdftitle={Title},pdfauthor={Author}}

%%%%%%%%%%%%%%%%%%%%%%%%%%%%%%%%%%%%%%%%%%%%%%%%%%%%%%%%%%%%%%%%%%%%%%%%%%%%%%%%
% Configuração das citações
%%%%%%%%%%%%%%%%%%%%%%%%%%%%%%%%%%%%%%%%%%%%%%%%%%%%%%%%%%%%%%%%%%%%%%%%%%%%%%%%
\usepackage[brazilian,hyperpageref]{backref}	 % Paginas com as citações na bibl
\usepackage[alf,
versalete,
abnt-emphasize = bf, % destaca o titulo em negrito;
abnt-etal-list = 3, % trabalhos com mais de 3 autores recebem et al.,;
abnt-etal-text = it, % escreve o et al., em italico;
abnt-and-type = &, % usa o carater '&' no lugar de 'e' para mais de um autor;
abnt-last-names = abnt, % trata sobrenomes 'estritamente' conforme a ABNT; e
abnt-repeated-author-omit = yes % autores com + de uma entrada recebem '____.'
]{abntex2cite}

% Configuração das referências bibliográficas
\renewcommand{\backref}{}
\renewcommand*{\backrefalt}[4]{	}


%%%%%%%%%%%%%%%%%%%%%%%%%%%%%%%%%%%%%%%%%%%%%%%%%%%%%%%%%%%%%%%%%%%%%%%%%%%%%%%%
% Informações da capa e da folha de rosto
%%%%%%%%%%%%%%%%%%%%%%%%%%%%%%%%%%%%%%%%%%%%%%%%%%%%%%%%%%%%%%%%%%%%%%%%%%%%%%%%

\titulo{TITULO DO PROJETO}

\autor{NOME DO AUTOR}

\local{Barra do Bugres - MT}

\data{2019}

% Alterar o nome do campus e do curso, caso houver necessidade
%\instituicao{
%	 Universidade Estadual do Estado de Mato Grosso -- UNEMAT\\
%	Faculdade de Ciências Exatas e Tecnológicas – FACET \\
%	Coordenação de Ciência da Computação – CCC
%}

\tipotrabalho{TCC}

% Informar {Titulação}{Nome}
\orientador{NOME DO PROFESSOR ORIENTADOR}

% Alterar o preâmbulo conforme necessário
\preambulo{Trabalho de conclusão de curso
	apresentado ao curso de Ciência da
	Computação da Universidade do Estado de
	Mato Grosso – UNEMAT, como requisito
	parcial para obtenção do grau de Bacharel.
}
% ---

% ---
% Configurações de aparência do PDF final


%%%%%%%%%%%%%%%%%%%%%%%%%%%%%%%%%%%%%%%%%%%%%%%%%%%%%%%%%%%%%%%%%%%%%%%%%%%%%%%%
% Outras configurações
%%%%%%%%%%%%%%%%%%%%%%%%%%%%%%%%%%%%%%%%%%%%%%%%%%%%%%%%%%%%%%%%%%%%%%%%%%%%%%%%

% Configuração da geração do PDF
\makeatletter
\hypersetup{
     	%pagebackref=true,
		pdftitle={\@title}, 
		pdfauthor={\@author},
    	pdfsubject={\imprimirpreambulo},
	    pdfcreator={LaTeX with abnTeX2},
		pdfkeywords={abnt}{latex}{abntex}{abntex2}{projeto de pesquisa}, 
		colorlinks=false,
		bookmarksdepth=4,
		pdfborder={0 0 0},
}
\makeatother


% O tamanho do parágrafo é dado por:
\setlength{\parindent}{1.3cm}

% Controle do espaçamento entre um parágrafo e outro:
\setlength{\parskip}{0.2cm}  % tente também \onelineskip

% compila o indice
\makeindex


%%%%%%%%%%%%%%%%%%%%%%%%%%%%%%%%%%%%%%%%%%%%%%%%%%%%%%%%%%%%%%%%%%%%%%%%%%%%%%%%
% CORPO DO TRABALHO ...
%%%%%%%%%%%%%%%%%%%%%%%%%%%%%%%%%%%%%%%%%%%%%%%%%%%%%%%%%%%%%%%%%%%%%%%%%%%%%%%%
\begin{document}

\selectlanguage{brazil}

% Retira espaço extra obsoleto entre as frases.
\frenchspacing 

% Inicializa a parte pre-textual
\pretextual

% Imprime a capa
\imprimircapa

% Imprime a folha de rosto
\imprimirfolhaderosto

% ------------------------------------------------------------------------------
% LISTA DE FIGURAS (Não altere nada aqui)
% ------------------------------------------------------------------------------
\pdfbookmark[0]{\listfigurename}{lof}
\listoffigures*
\cleardoublepage


% ------------------------------------------------------------------------------
% LISTA DE TABELAS (Não altere nada aqui)
% ------------------------------------------------------------------------------
\pdfbookmark[0]{\contentsname}{lot}
\listoftables*
\cleardoublepage

% ------------------------------------------------------------------------------
% LISTA DE SIGLAS E ABREVIATURAS
% ------------------------------------------------------------------------------
% Edite a lista de siglas conforme o modelo abaixo
\begin{siglas}
	\item [SIGLA] {DESCRIÇÃO DA SIGLA}
	% Incluir as siglas aqui ...
\end{siglas}


% ------------------------------------------------------------------------------
% SUMÁRIO (Não altere nada aqui)
% ------------------------------------------------------------------------------
\pdfbookmark[0]{\contentsname}{toc}
\tableofcontents*
\cleardoublepage

% ------------------------------------------------------------------------------
% ELEMENTOS TEXTUAIS
% ------------------------------------------------------------------------------
%PROJETO DE PESQUISA
% Introduz a parte textual
\textual


\chapter{TEMA}
TEMA DO PROJETO
	
\chapter{DELIMITAÇÃO DO TEMA}	
	DELIMITAÇÃO
	
\chapter{PROBLEMA}
DIZER QUAIS AS PROBLEMÁTICAS
	
\chapter{HIPÓTESE}
O PORQUE DISSO ACONTECER
	
\chapter{OBJETIVOS}
	\section{Objetivo geral}
	OBJETIVO DO PROJETO
	
	\section{Objetivos específicos}
		\begin{enumerate}
			\item MOTIVO 1;
			\item MOTIVO 2;
			\item MOTIVO 4; ...
			\end{enumerate}

\chapter{JUSTIFICATIVA}
JUSTIFICAR O DO PORQUE ESTA FAZENDO ESTE PROJETO \cite{mota2012descobrindo}.

		\begin{citacao}
			CITACAO DIRETA
		\end{citacao}


\chapter{FUNDAMENTAÇÃO TEÓRICA}
	\section{\textit{Linux}}
	
	FUNDAMENTAÇÃO TEÓRICA \cite{nomenacitacao}.
	
	
	
		\begin{citacao}
		CITAÇÃO DIRETA
		\end{citacao}
	
	
		\subsection{SUBSEÇÃO}
		
			\begin{figure*}[h]
			\caption{\label{Debian}\textit{QRcode Github}}
			\centering
			{\includegraphics[width=9cm]{Imagens/Karan_GitHub.png}}
			\legend{Fonte: MODELO PARA ADICAO DE IMAGENS}
			\end{figure*}
	

	
\chapter{Procedimentos metodológicos}
	
	

\chapter{Cronograma}
\begin{table}[htb]
%\ABNTEXfontereduzida
\caption[Cronograma]{Cronograma.}
\label{tab-nivinv}		
\scalefont{0.7}
	\begin{tabular}{|c|c|c|c|c|c|c|c|c|c|c|c|}
		\hline
		\textbf{Ano} & \multicolumn{11}{c|}{\textbf{-------------------- 2019/1 --------------------------|------------------- 2019/2 -------------------}} \\ \hline
		Atividades | MÊS & Fev. & Mar. & Abr. & Mai. & Jun. & Jul. & Ago. & Set. & Out. & Nov. & Dez \\ \hline
		\begin{tabular}[c]{@{}c@{}}Desenvolvimento do Tema e \\ Objetivos\end{tabular} &  &  &  &  &  &  &  &  &  &  &  \\ \hline
		\begin{tabular}[c]{@{}c@{}}Desenvolvimento do Problema,\\ Hipótese e Justificativa\end{tabular} &  &  &  &  &  &  &  &  &  &  &  \\ \hline
		\begin{tabular}[c]{@{}c@{}}Desenvolvimento da\\ Metodologia e Fundamentação Teórica\end{tabular} &  &  &  &  &  &  &  &  &  &  &  \\ \hline
		\begin{tabular}[c]{@{}c@{}}Desenvolvimento do\\ Cronograma e Orçamento\end{tabular} &  &  &  &  &  &  &  &  &  &  &  \\ \hline
		\begin{tabular}[c]{@{}c@{}}Encontros\\ de Orientação\end{tabular} & X & X & X & X & X & X & X & X & X & X & X \\ \hline
		\begin{tabular}[c]{@{}c@{}}Defesa do\\ Projeto de Pesquisa\end{tabular} &  &  &  &  &  &  &  &  &  &  &  \\ \hline
		\begin{tabular}[c]{@{}c@{}}Desenvolvimento\\ da Introdução\end{tabular} &  &  &  &  &  &  &  &  &  &  &  \\ \hline
		\begin{tabular}[c]{@{}c@{}}Desenvolvimento\\ do TCC\end{tabular} &  &  &  &  &  &  &  &  &  &  &  \\ \hline
		\begin{tabular}[c]{@{}c@{}}Correção de\\ Erros\end{tabular} &  &  &  &  &  &  &  &  &  &  &  \\ \hline
		\begin{tabular}[c]{@{}c@{}}Elaboração da\\ apresentação do TCC\end{tabular} &  &  &  &  &  &  &  &  &  &  &  \\ \hline
		\begin{tabular}[c]{@{}c@{}}Apresentação\\ do TCC\end{tabular} &  &  &  &  &  &  &  &  &  &  &  \\ \hline
		\begin{tabular}[c]{@{}c@{}}Entrega Versão\\ final\end{tabular} &  &  &  &  &  &  &  &  &  &  &  \\ \hline
	\end{tabular}
%\legend{Fonte: Karan}
\end{table}

\chapter{ORÇAMENTO}
\begin{table}[!h]
\caption[Orcamento]{Orçamento.}
\label{tab-nivinv}
\scalefont{1.2}	
	\begin{tabular}{|c|c|c|}
		\hline
		\textbf{Descrição das Despesas} & \textbf{Quantidade} & \textbf{Valor Estimado (em reais)} \\ \hline
		Fotocópia & 200 & R\$ 100,00 \\ \hline
		Encadernação & 7 & R\$ 50,00 \\ \hline
		Aquisição de bibliografias & 2 & R\$ 60,00 \\ \hline
		Aquisição de cursos & 2 & R\$ 100,00 \\ \hline
		\textbf{Total} &  & \textbf{R\$ 310,00} \\ \hline
	\end{tabular}
\end{table}

% Finaliza o bookmarking do PDF
\phantompart

% ------------------------------------------------------------------------------
% ELEMENTOS PÓS-TEXTUAIS
% ------------------------------------------------------------------------------

% Introduz a parte pós-textual
\postextual

% Insere as referências bibliográficas
\bibliography{bibliografia}

\end{document}
